% HGG, 2021-11-17 Korrektur gelesen
\documentclass[a4paper,11pt]{article}
\usepackage[ngerman]{babel} % Sprache festlegen
\usepackage[utf8]{inputenc}
\usepackage{a4wide,url,graphicx,enumitem} 

\title{SDG in deutschen Bundestagsreden\\[12pt] \large Eine Distant Reading
  Analyse mittels Machine Learning}

\author{Veronika Marie Heuten}

\date{31. Oktober 2021}

\setlist{noitemsep}
\parindent0pt
\parskip3pt

\begin{document}
\maketitle
\vfill
\thispagestyle{empty}
\tableofcontents
\clearpage

\section{Einleitung}
Die Menschheit steht vor vielen schwerwiegenden Problemen. Noch immer leben
viele Menschen in Armut, hungern, haben keinen Zugang zu sauberem Trinkwasser
und arbeiten unter menschenunwürdigen Bedingungen. Ein Großteil dieser
Probleme resultiert aus der Art, wie wir Menschen miteinander leben, aber auch
mit der Umwelt umgehen und welche Hierarchien als selbstverständlich angesehen
werden. Diese sind durch Ungleichheit zwischen Geschlechtern, Nationen und
Gesellschaftsschichten, die der Mensch im Laufe seines Bestehens selbst
hervorgebracht hat, entstanden. Dazu kommt, dass diese Entwicklung in den
größten Teilen irreversibel ist. Viele dieser Missstände bedingen sich
gegenseitig und verschlimmern einander. So geht Armut beispielsweise mit dem
unzureichenden Zugang zu Trinkwasser einher, was wiederum hygienische
Missstände verursacht und Krankheiten begünstigt. Kranke Menschen wiederum
können nur schwer einer Arbeit nachgehen, was wiederum auch Armut und Hunger
begünstigt.

Als wären diese Umstände nicht schon aussichtslos genug, traf im Jahr 2020 ein
zusätzliches Problem auf die Weltgemeinschaft. Mit der Coronapandemie
existiert nun ein weiteres Problem, das nur global unter Kontrolle gebracht
werden kann. Durch den Ausbruch der pandemischen Lage von globaler Tragweite
wurden die bis dahin essentiellen Probleme der Menschen in den Hintergrund
gerückt. Der Kampf gegen den Klimawandel, Zerstörung von Ökosystemen, Hunger,
Armut und Ungerechtigkeit musste warten. Und das, obwohl viele dieser Themen
durch die Ausnahmesituation verstärkt wurden. Besonders im globalen Süden
wurden viele Probleme durch die pandemische Lage noch verstärkt. Wie soll die
Menschheit nun gegen diese Probleme ankommen? Die Art zu leben des globalen
Nordens und dessen Einfluss auf die Umwelt lassen sich nicht rückgängig
machen.  Die einzige Möglichkeit, allen Menschen ein Leben in Wohlstand,
Sicherheit und Gesundheit zu bieten, liegt darin, einen Wandel hin zu
nachhaltigem Wirtschaften auf den Weg zu bringen.

In der vorliegenden Arbeit soll untersucht werden, wie sich die deutsche
Politik mit den 17 Sustainable Development Goals der Vereinten Nationen
beschäftigt. Zunächst wird ein kurzer Überblick über die Entstehungsgeschichte
der Sustainable Development Goals und ihren Inhalt gegeben. Im Anschluss daran
wird erläutert, wie in dieser Analyse vorgegangen wurde und welche Methoden
zum Einsatz kamen. Dann folgt ein Beschreibung der verwendeten Datensätze und
wie mit diesen verfahren wurde. In einem weiteren Abschnitt sind die
Ergebnisse präsentiert und anschließend interpretiert. Im abschließenden Teil
wird das Vorgehen der Analyse reflektiert und ein Fazit gezogen.
%\clearpage

\section{Sustainable Development Goals}
Im Jahr 2000 verständigten sich die Vereinten Nationen auf einem UN-Gipfel
dazu, den elementaren Problemen der Menschheit gemeinschaftlich
entgegenzuwirken. Daraufhin wurden die sogenannten \emph{Millenium Development
  Goals} (MDGs) verabschiedet. In diesen wurden acht Ziele festgelegt, die bis
zum Jahr 2015 erreicht werden sollten. Diese Ziele wurden wie folgt definiert
\cite{BMZ}:

\begin{enumerate}
\item den Anteil der Weltbevölkerung halbieren, der unter extremer Armut und
  Hunger leidet
\item allen Kindern eine Grundschulausbildung ermöglichen
\item die Gleichstellung der Geschlechter fördern und die Rechte von Frauen
\item die Kindersterblichkeit verringern
\item die Gesundheit der Mütter verbessern
\item HIV/Aids, Malaria und andere übertragbare Krankheiten bekämpfen
\item den Schutz der Umwelt verbessern
\item eine weltweite Entwicklungspartnerschaft aufbauen
\end{enumerate}

Durch diese ehrgeizigen und bis dahin in der Geschichte einmaligen Ziele
konnten sich in den 15 Jahren des Geltungszeitraumes der MDGs mehr als eine
Milliarde Menschen aus extremer Armut befreit, wurde Hunger abgebaut, und so
vielen Mädchen und Frauen wie nie zuvor eine Schulbildung ermöglicht. Jedoch
wurden diese Ziele bis zum Jahr 2015 nicht erreicht, was den
UN-Generalsekretär Ban Ki-Moon dazu brachte eine
\emph{Post-2015-Entwicklungsagenda} zu fordern, die diese Ziele weiter
vorantreiben soll \cite{UN}.

In der Tradition der MDGs wurde im September 2015 die \emph{Agenda 2030 für
  nachhaltige Entwicklung} mit den 17 darin enthaltenen \emph{Sustainable
  Development Goals} (SDGs) verabschiedet. Diese Ziele sind die erste
internationale Übereinkunft, die das Prinzip der Nachhaltigkeit mit der
Bekämpfung von Hunger und Armut mit einander in Beziehung setzt \cite{BMZ}.
Die Vereinten Nationen haben es sich zur Aufgabe gemacht, eine sozial,
ökonomisch und ökologisch gerechte Welt bis zum Jahre 2030 zu erreichen.

Die 17 Ziele lauten in ihren Überschriften \cite{Agenda2030}:
\begin{enumerate}
\item Keine Armut
\item Kein Hunger 
\item Gesundheit und Wohlergehen 
\item Hochwertige Bildung 
\item Geschlechtergleichheit 
\item Sauberes Wasser und Sanitäreinrichtungen 
\item Bezahlbare und sauberer Energie 
\item Menschenwürdige Arbeit und Wirtschaftswachstum 
\item Industrie, Innovation und Infrastruktur
\item Weniger Ungleichheit (in und zwischen Ländern)
\item Nachhaltige Städte und Gemeinden
\item Nachhaltiger Konsum und Produktion 
\item Maßnahmen zum Klimaschutz
\item Leben unter Wasser (erhalten und nachhaltig nutzen)
\item Leben an Land (schützen, wiederherstellen)
\item Frieden, Gerechtigkeit und starke Institutionen 
\item Partnerschaften zur Erreichung der Ziele  
\end{enumerate}

Diese 17 Ziele lassen sich in fünf etwas gröberen Handlungsfeldern
zusammenfassen: \textbf{Mensch} (People), \textbf{Planet} (Planet),
\textbf{Wohlstand} (Prosperity), \textbf{Frieden} (Peace) und
\textbf{Partnerschaft} (Partnership).

Auch wenn in den letzten Jahren Fortschritte gemacht wurden, wurde das
Erreichen der Ziele durch die Coronapandemie weit zurückgeworfen. Darunter
leiden besonders die ärmsten Länder \cite{Agenda2030}. Im September 2019 wurde
festgestellt, dass die Entwicklungsziele mit der bisherigen Strategie bis 2030
nicht erreicht werden können. Um dies auszugleichen und die SDGs doch noch bis
2030 zu erreichen, müsste die Weltgemeinschaft jährlich zusätzlich 4,5
Billionen Euro investieren. Besonders durch die weltweite pandemische Lage
wurde den Vereinten Nationen vor Augen geführt, dass zur Bewältigung von
globalen Herausforderungen nachhaltiges Handeln von besonderer Wichtigkeit ist
\cite{Bericht}.

Gerade unter diesem Gesichtspunkt stellt sich die Frage, wie die SDGs im
politischen Diskurs des deutschen Bundestages thematisiert werden.
%\clearpage
 
\section{Erläuterung des Vorgehens}
Um der Beantwortung der Frage nach der Präsenz der SDGs im Diskurs der
deutschen Politik näher zu kommen, soll dies im Folgenden mittels Distant
Reading untersucht werden. Distant Reading bezeichnet die eher quantitativ
geprägte Untersuchung von großen Mengen an Textdaten. Hierbei werden zum
Beispiel Worthäufigkeiten oder Kookkurenzen interpretiert. Dies steht im
Gegensatz zur traditionellen Methode des Close Reading, worunter man die
herkömmliche inhaltliche Interpretation von Texten versteht \cite{Moretti}.

\subsection{Was ist Topic Modeling und wie funktioniert es?}
Die Methode des Topic Modeling erstellt für eine Sammlung von Texten oder
innerhalb eines einzelnen Textes verschiedene „Topics“ -- auf Deutsch:
„Überschriften“. Diese Topics bestehen sich aus einer Menge von Worten, die
häufig zusammen auftauchen. Hier wird allein auf Grund der Häufigkeit, mit der
die Worte gemeinsam auftauchen, davon ausgegangen, dass diese in thematischem
Bezug zueinander stehen. Die Topics werden aus den analysierten Texten heraus
gefunden und benötigen keine externen Wörterbücher oder Trainingsdatensätze.

\subsection{Verwendete Daten und Vorgehen}
Um der Forschungsfrage nachzugehen, inwieweit die 17 Ziele der nachhaltigen
Entwicklung in der deutschen Politik beachtet werden, wird zum einen ein
Korpus aus den Volltexten der Plenarsitzungen des deutschen Bundestages
verwendet. Der deutsche Bundestag stellt alle Ple\-nar\-protokolle und
Drucksachen ab der ersten Wahlperiode sowie die Biografien aller Abgeordneten
seit 1949 und Abstimmungslisten aller namentlichen Abstimmungen öffentlich zur
Verfügung \cite{Bundestag}. Jedoch sind nicht alle diese veröffentlichten
Dokumente maschinenlesbar. Das gemeinnützige Projekt \emph{Open Discourse} hat
daher aus diesen öffentlich zugänglichen Daten einen Korpus erstellt, der alle
Plenarsitzungen des deutschen Bundestags inklusive Metadaten (Datum,
Sprecher*in, Wahlperiode) enthält \cite{OpenDiscourse}. Dieser Korpus ist frei
zugänglich und wurde im Rahmen dieser Arbeit verwendet. Für die Analyse wurden
alle Redebeiträge seit dem 01.01.2016 einbezogen, da seit diesem Tag die SDGs
offiziell in Kraft getreten sind.

Auf der anderen Seite wurde ein weiteres Modell mit anderen Texten trainiert
um herauszufinden, ob sich einzelne SDGs überhaupt als Topic ausmachen lassen.
Dazu wurden alle Publikationen des BNE-Forums heruntergeladen und als Korpus
verwendet. Das BNE-Forum ist das \emph{Forum für Bildung für nachhaltige
  Entwicklung} des Bundesministeriums für Bildung und Forschung. Um eine
friedliche und nachhaltige Gesellschaft zu gestalten, wie es sich die SDGs als
Ziel gesetzt haben, ist nachhaltige Bildung unabdingbar. So wird Bürger*innen
die Fähigkeit an die Hand gegeben, vorausschauend zu denken,
interdisziplinäres Wissen zu verinnerlichen, autonom zu handeln und an
gesellschaftlichen Entscheidungsprozessen teilzuhaben~\cite{BNE}. Da in den
Veröffentlichungen des BNE die SDGs mit Sicherheit thematisiert werden, eignet
sich dieses Korpus sehr gut als Vergleich zu den Bundestagsreden um zu
überprüfen, ob sich die in den mittels LDA extrahierten Topics aus den Texten
wiedererkennen lassen oder ob der Ansatz nicht für die Klärung der
Forschungsfrage geeignet ist.

Um die Topics aus den Korpora zu erhalten wird MALLET verwendet. MALLET ist
ein Java basiertes Package, das mit dem Little Mallet Wrapper auch für Python
verfügbar ist. Das MAchine Learning LanguagE Toolkit wurde von Andrew McCallum
entwickelt. Er lehrt an der University of Massachusets Amherst \cite{MALLET,
  MALLET_WELSH}.

Das MALLET Topic Modelling ist LDA basiert. LDA steht für „Latent Dirichlet
Alloca\-tion“, hierbei handelt es sich um ein im Jahr 2000 vorgestelltes
Wahrscheinlichkeitsmodell. Die betrachteten Worte aus den zu verarbeitenden
Texten werden gruppiert und ergeben die Topics~\cite{LDA}.

Im folgenden Modell wurden für jeden Korpus 17 Topics generiert, um im
Idealfall ein 1:1 Matching mit den 17 SDGs zu bekommen.

Die gewonnenen Topics werden dann betrachtet und durch Ermessen der Autorin
den SDGs zugeordnet. Die Benennung von Topics aufgrund ihres Inhaltes durch
die durchführenden Forscher*innen ist bei ungelabelten LDA gängige Praxis
\cite{Ramage}.  Im Prozess der Datenverarbeitung wird zunächst jegliche Art von
Großschreibung entfernt, sodass alle Wörter klein geschrieben sind. Danach
werden alle Satzzeichen und Zahlen aus dem Korpus entfernt. Der entscheidende
Teil für eine Korpusanalyse ist die Entfernung aller Stoppwörter. Stoppwörter
sind Worte, die sehr häufig auftreten und keinen Informationsgewinn für das
Ergebnis bedeuten. Solche Listen können vorgerfertigt verwendet und manuell
ergänzt werden. In der vorliegenden Analyse wird die deutsche Stoppwortliste
des Python \texttt{nltk} Packages verwendet.
%\clearpage

\section{Ergebnisse}
\subsection{Metadaten Korpus}
Das Korpus der Bundestagsreden seit dem 01.01.2016 enthält 4028113 Wörter und
ein Vokabular von 171404 verschiedenen Wörtern. Das BNE-Korpus besteht aus 35
Publikationen, enthält insgesamt 365470 Wörter und hat ein Vokabular von 42658
Wörtern. Das Korpus mit den Bundestagsreden ist in etwa elf Mal so groß wie
das BNE-Korpus.

\subsection{Ergebnisse des Topic Modellings} 
Nachfolgend sind die aus der Analyse hervorgegangenen Topics aufgezeigt. In
der untersten Zeile werden die SDGs angezeigt, die am ehesten mit den Tpoics
in Verbindung stehen könnten. Dies ist Ergebnis einer inhaltlichen
Interpretation der durch das Topic Modeling erhaltenen Informationen.

\subsubsection{Topic Modeling BNE-Korpus}

\paragraph{Topic 1} 0,22632\\
dekade unesco deutschland aktivitäten nationalkomitee wurde projekte deutschen
jahre konferenz tisch umsetzung deutsche aktionsplan erklärung wurden bne bonn
runden projekt  \\ 
\textbf{Zugehörige SDG:} 17. Partnerschaften zur Erreichung der Ziele

\paragraph{Topic 2} 0,0547\\
nachhaltigkeit heute unesco menschen kultur ressourcen müssen leben staaten
vielfalt politik verstehen länder konvention menschenrechte begriff
gesellschaften wachstum kulturelle bevölkerung   \\ 
\textbf{Zugehörige SDG:} keine Übereinstimmung 

\paragraph{Topic 3} 0,63499 \\
bne sowie nachhaltigkeit akteure rahmen ziele bereich nachhaltigen engagement
hochschulen kommunen zusammenarbeit akteuren unterstützung maßnahmen
vernetzung schule aktivitäten entwickelt arbeit   \\ 
\textbf{Zugehörige SDG:} 4. Hochwertige Bildung; 17. Parternerschaften zur
Erreichung der Ziele 

\paragraph{Topic 4} 0,3658 \\
bne unesco sowie nachhaltigen entwicklung akteure dekade umsetzung
weltaktionsprogramm wap ebene aktivitäten unterstützung bereich handlungsfeld
weltaktionsprogramms rolle bildung rahmen wichtige    \\ 
\textbf{Zugehörige SDG:} keine Übereinstimmung
  
\paragraph{Topic 5} 0,16238 \\
menschen natur info handeln kasten foto lebens aspekte weltweit kriterien
natürlichen erde beispiele verändern gruppe generationen http
qualitätskriterien verhalten www   \\ 
\textbf{Zugehörige SDG:} keine Übereinstimmung
    
\paragraph{Topic 6} 0,07403 \\
indikatoren entwicklung nachhaltige bildung indikator wurden bzw bereich daten
wurde forschung hinsichtlich unece anzahl bildungsberichterstattung
indikatorenset ergebnisse bildungsstandards sowie hochschulen    \\
 \textbf{Zugehörige SDG:} 4. Hochwertige Bildung 
 
 
\paragraph{Topic 7} 0,00887 \\
vielfalt biologische natur biologischer biologischen sowie vgl nutzung schutz
abb jugendlichen wald bedeutung projekt umwelt schlüsselthemen deutschland
arten kinder querbeet    \\ 
\textbf{Zugehörige SDG:} 13. Maßnahmen zum Klimaschutz; 15. Leben an Land 

\paragraph{Topic 8} 0,04604 \\
unesco deutschland deutschen kultur vielfalt deutsche kommission inklusive
prof ziele kulturweit kulturerbe welterbe fördern kulturelle netzwerk erhalten
projekte oer weltweit   \\ 
\textbf{Zugehörige SDG:} 17. Partnerschaften zur Erreichung der Ziele 

\paragraph{Topic 9} 0,05341  \\
mobilität thema bildungsbereich titel broschüre herausgeber sekundarstufe
lernmedium erscheinungsjahr zukunft außerschulische teilnehmer umwelt
multiplikatoren primarstufe website verkehr link online dekade    \\ 
\textbf{Zugehörige SDG:} 4. Hochwertige Bildung 

\paragraph{Topic 10} 0,8289 \\
mehr menschen gibt wurde welt wissen immer viele themen geht neue jahr
nachhaltigen jahren wurden seit etwa eigenen gestalten beispiel    \\
\textbf{Zugehörige SDG:} keine Übereinstimmung 

\paragraph{Topic 11} 1,22198 \\
entwicklung bildung nachhaltige unesco lernen umsetzung nachhaltigkeit
deutschland schulen nachhaltiger gesellschaft forschung kommission umwelt
dabei deutschen berlin deutsche internationale handeln \\
\textbf{Zugehörige SDG:} 4. Hochwertige Bildung 
   
\paragraph{Topic 12} 0,07148 \\
hochschulen nachhaltigkeit universität hochschule lehre forschung nachhaltige
netzwerk studierenden http netzwerks www entwicklung studierende uni leuphana
sowie seit universitäten betrieb \\
\textbf{Zugehörige SDG:} 4. Hochwertige Bildung 
   
\paragraph{Topic 13} 0,09144 \\
bne ziel handlungsfeld hochschule iii umsetzung nationalen kommunen fachforum
aktionsplan bmbf commitment fördert länder bildung non maßnahmen schule
hochschulen nap    \\ 
\textbf{Zugehörige SDG:} 4. Hochwertige Bildung 

\paragraph{Topic 14} 0,16984 \\
dekade projekte projekten bne projekt verbreitung nap entwicklung laufenden
abb verankerung ergebnisse aktivitäten maßnahmen zielgruppen gruppe eher vgl
zahl trifft    \\ 
\textbf{Zugehörige SDG:} keine Übereinstimmung 
    
\paragraph{Topic 15} 0,05064 \\
stärken strukturen netzwerk nachhhaltig thüringen spohns hsc gelsenkirchen
haus klimahaus budde tübingen neumarkt kita till lernorte prof uelzen stadt  \\
\textbf{Zugehörige SDG:} 11. Nachhaltige Städt und Gemeinden 
    
\paragraph{Topic 16} 0,05569 \\
hte and education development sustainable that are this with germany have not
decade berlin all which our can european national\\ 
\textbf{Zugehörige SDG:} keine Übereinstimmung 

\paragraph{Topic 17} 0,11911\\
stadt dekade projekt kommunen wurde bne zukunft schulen evaluation
ansprechpartner aktivitäten bayern weitere initiative umweltbildung beispiel
informationen netzwerk reichweite kinder    \\ 
\textbf{Zugehörige SDG:} 4. Hochwertige Bildung

Inhaltliche Zuordnung zu SDGs: 
\begin{center}
\begin{tabular}{ |c|c|} \hline
Nummer und Bezeichnung SDG & inhaltlich einem Topic zugeordnet\\\hline
1. Keine Armut & -- \\\hline 
2. Kein Hunger & -- \\\hline
3. Gesundheit und Wohlergehen & -- \\\hline
4. Hochwertige Bildung & 6 mal \\\hline 
5. Geschlechter Gerechtigkeit & -- \\\hline
6. Sauberes Wasser und Sanitäreinrichtungen & -- \\\hline
7. Bezahlbare und saubere Energie & -- \\\hline 
8. Menschenwürdige Arbeit und Wachstum & -- \\\hline 
9. Industrie, Innovation und Infrastruktur & -- \\\hline 
10. Weniger Ungleichheit & -- \\\hline 
11. Nachhaltige Städte und Gemeinden & 1 mal \\\hline 
12. Nachhaltiger Konsum und Produktion & -- \\\hline 
13. Maßnahmen zum Klimaschutz & 1 mal \\\hline 
14. Leben unter Wasser & -- \\\hline 
15. Leben an Land & 1 mal \\\hline 
16. Frieden, Gerechtigkeit und starke Institutionen & -- \\\hline 
17. Partnerschaften zur Erreichung der Ziele & 3 mal \\\hline 
\end{tabular}
\end{center}

\subsubsection{Topic Modeling Bundestagsreden}
Number of Documents: 1\\
Mean Number of Words per Document: 4028113.0\\
Vocabulary Size: 171404\\

\paragraph{Topic 1} 2.869,3306 \\
schön ziel übrigens zweitens bekommen lage erteile politische beiden vorgelegt
pflege jemand rechte weder form schützen gezeigt entwurf aktuell wünschen\\ 
\textbf{Zugehörige SDG:}  keine Übereinstimmung

\paragraph{Topic 2} 224,92371 \\
spielball laufzeit partnerländer laufende wirtschaftsförderung wladimir
aufgestellten dadurch berufsbildungsbericht ausbauziele einzelpersonen
partizipieren projekt kitaplus steile nüchtern billiger opel gepflegt
grundgesetz  \\
\textbf{Zugehörige SDG:} 8. Menschenwürdige Arbeit und Wirtschaftswachstum 

\paragraph{Topic 3} 219,32355 \\
zurückzunehmen darzulegen lebhaft kulturstaatsministerin mullahs
antibiotikaresistenzen unruhig rüstungskontrolle ultima angrenzenden feindbild
fernen abscheuliche spr kirchlichen niedrigem unb eingeordnet ahne
infektionsschutz   \\ 
\textbf{Zugehörige SDG:} 3. Gesundheit und Wohlergehen 
    
\paragraph{Topic 4} 220,99597 \\
umg dsgvo totaler flankieren menschenrechtskonvention entstandenen volker
verletzten subvention passend palette rentenbeiträge nachfolge ausdrücklichen
allerletzten heimat prophezeit hautnah misstrauen pantel    \\ 
\textbf{Zugehörige SDG:} keine Übereinstimmung    

\paragraph{Topic 5} 14.859,18463 \\
herr vielen kollegen kolleginnen fraktion frau dank müssen mehr heute menschen
herren geht bundesregierung präsident damen wort kollege spd mal    \\
\textbf{Zugehörige SDG:} keine Übereinstimmung
   
\paragraph{Topic 6} 10.499,17592 \\
liebe schon damen dank kollegin sagen herren frage kollege dafür ganz deshalb
gesagt fdp euro gesetz deutlich worden unserer zeit    \\
\textbf{Zugehörige SDG:} keine Übereinstimmung   

\paragraph{Topic 7} 567,1588 \\
ehemaligen belastet japan teurer änderungsanträgen künstlich unterstrichen
erkenntnissen pay berichterstatterin einwanderungsgesetz best hauptaufgabe
meier umfangreiches artur dieselskandal gesetzgeberisch wasserstraße test   \\
\textbf{Zugehörige SDG:} keine Übereinstimmung
  
\paragraph{Topic 8} 224,69176 \\
erhofft heranziehen kompromisses neun künstlerischen basierend schlau
forschungsinstitut finanzmarktnovellierungsgesetz wirtschaftsforschung
begrenzt familienarmut informationelle ressortübergreifende kriegerischen
resolve widerfahren herausgeholt familienausschuss landshut \\ 
\textbf{Zugehörige SDG:} keine Übereintimmung
    
\paragraph{Topic 9} 227,04995 \\
zuwendung erp versuchte gelebter sportbund trainingsmission
friedensverhandlungen informieren bmel fahrgastzahlen wiege pädagogisch
gemachte ertrunken trivial asylsystems haushalt relativierung förder
hochverehrte \\
\textbf{Zugehörige SDG:} keine Übereinstimmung

\paragraph{Topic 10} 226,863 \\
traumatisiert hinzunehmen wohnung vertragsparteien psychologische
genossenschaften konkret weiterarbeiten genehmigten vertrages stasiunterlagen
hehre eier einwanderungsgesellschaft abgleich gegossen lebensbereiche
mitarbeiterin bittsteller ungesunde   \\ 
\textbf{Zugehörige SDG:} keine Übereinstimmung

\paragraph{Topic 11} 220,84211 \\
geburtenrate ausbreiten kartellbehörden freundin halbzeit feindbilder
miterlebt leichtere absprache produ gesagten mitzuhelfen groteske
entwicklungsausgaben geschockt staatliches zähle vorzuhalten
gesundheitsschäden beide  \\
\textbf{Zugehörige SDG:} keine Übereinstimmung 
    
\paragraph{Topic 12} 1.668,70828 \\
jahre europäischen wurden aufmerksamkeit antwort weiteren wegen fördern
demokraten besteht europas freiheit irgendwie zahlreiche verhindert besseren
betrachten anfrage ursachen staatsminister  \\ 
\textbf{Zugehörige SDG:} keine Übereinstimmung
    
\paragraph{Topic 13} 220,67603 \\
ressortabstimmung darstellt pass herunter eingeständnis ablegen ressource
pflanze netzwerkdurchsetzungsgesetzes angestoßen mvz beabsichtigen
bundesgesundheitsministeriums bewahrt kleinreden prostituiertenschutzgesetz
aufbaut koordinator einwenden mieterschutz  \\ 
\textbf{Zugehörige SDG:} keine Übereinstimmung

\paragraph{Topic 14} 5.517,16359 \\
vielleicht einfach heißt minister stehen wer unternehmen maßnahmen frauen
ersten angesprochen zeigt allerdings bündnis ländern führen ganze fordern
bleiben schließe   \\ 
\textbf{Zugehörige SDG:} keine Übereinstimmung 
   
\paragraph{Topic 15} 4.533,16673 \\
linke tun leider zukunft bund insbesondere sicherheit lange verantwortung
beratung nächsten fast sogar weniger kommunen abgeordneten diskutieren übrigen
tat sowie   \\ 
\textbf{Zugehörige SDG:} keine Übereinstimmung 
    
\paragraph{Topic 16} 9.211,67119 \\
gibt deutschland gut gerade antrag cdu/csu frau nächster wichtig prozent
richtig präsidentin sage immer millionen sagen kolleginnen frage union
milliarden  \\
\textbf{Zugehörige SDG:} keine Übereinstimmung 
     
\paragraph{Topic 17} 224,4915 \\
kompetent festlegungen unwürdigen bauliche vollendung kooperativen
verfassungsgemäß wasserstofftechnologie ausfluss prosperierenden sträflich
kritische fbb heut steuerbetrug interessenausgleich passwörter überzogenen
sozialbereich produzierte \\ 
\textbf{Zugehörige SDG:} keine Übereinstimmung 
      
Inhaltliche Zuordnung zu SDGs: 
\begin{center}
\begin{tabular}{ |c|c|} \hline
Nummer und Bezeichnung SDG & inhaltlich einem Topic zugeordnet\\\hline
1. Keine Armut & -- \\\hline 
2. Kein Hunger & -- \\\hline
3. Gesundheit und Wohlergehen & 1 mal \\\hline
4. Hochwertige Bildung & --  \\\hline 
5. Geschlechter Gerechtigkeit & -- \\\hline
6. Sauberes Wasser und Sanitäreinrichtungen & -- \\\hline
7. Bezahlbare und saubere Energie & -- \\\hline 
8. Menschenwürdige Arbeit und Wachstum & 1 mal \\\hline 
9. Industrie, Innovation und Infrastruktur & -- \\\hline 
10. Weniger Ungleichheit & -- \\\hline 
11. Nachhaltige Städte und Gemeinden & -- \\\hline 
12. Nachhaltiger Konsum und Produktion & -- \\\hline 
13. Maßnahmen zum Klimaschutz & -- \\\hline 
14. Leben unter Wasser & -- \\\hline 
15. Leben an Land & 1 mal \\\hline 
16. Frieden, Gerechtigkeit und starke Institutionen & -- \\\hline 
17. Partnerschaften zur Erreichung der Ziele & -- \\\hline 
\end{tabular}
\end{center}

\subsection{Interpretation der Ergebnisse}
Die vorliegenden Ergebnisse zeigen leider kein eindeutiges Bild. Die Topics
sind diffus und lassen sich nur in den seltensten Fällen einem der 17 SDGs
zuordnen, wie in den Tabellen gezeigt ist. Werfen wir nun zuerst einen
genaueren Blick auf die Ergebnisse des BNE-Korpus. Hier sind die Topics
größtenteils dem SDG 4 „Hochwertige Bildung“ zuzuordnen. Dies ist angesichts
der Herkunft der Texte nicht wirklich verwunderlich. Immerhin stammen diese
aus dem Forum Bildung für nachhaltige Entwicklung. Dass die meisten Texte
demnach inhaltlich durch das Thema Bildung bestimmt werden, ist demnach klar.
Die anderen Topics, die sich im weitesten Sinne mit einem der SDGs in
Verbindung bringen lassen, sind die SDGs 11 „Nachhaltige Städte und
Gemeinden“, 13 „Maßnahmen zum Klimaschutz“, 15 „Leben an Land“ und 17
„Partnerschaften zur Erreichung der Ziele“, welches drei Mal mit den Topics in
Verbindung gebracht werden kann. Diese SDGs stehen auch in Verbindung zu den
Themen Klimaschutz, Umweltschutz und Nachhaltigkeit.  Humanitäre Themen wie
die Bekämpfung von Hunger und Armut, Geschlechtergerechtigkeit und Frieden
werden hingegen gar nicht erwähnt. Interessant ist auch Topic 16 der
BNE-Analyse. Hier erscheinen nur englische Wörter, was entweder darauf
hindeuten könnte, dass ein englischer Text in dem Korpus enthalten ist oder
alle englischen Zitate gruppiert wurden. Zusammenfassend lässt sich über das
Topic Modeling des BNE-Korpus sagen, dass sich einige SDGs ausmachen lassen,
aber nur jene, die einen Bezug zu Bildung im Bereich Nachhaltigkeitsthemen
haben.

Das Topic Modeling des Korpus der Bundestagsreden gestaltet sich in der
Interpretation sogar noch etwas schwieriger. Die Themen der einzelnen Topics
sind nicht immer auszumachen. Dies liegt mit Sicherheit auch daran, dass das
Korpus aus einem ganz anderen Kontext kommt. Hier sind alle Redebeiträge
wortwörtlich verschriftlicht, und Ziel der Analyse ist es herauszufinden, ob
die SDGs überhaupt thematisiert werden. Der Ursprung des Korpus führt schon
zum ersten Problem, dass auch solche Wörter wie „vielen“ „Dank“ „liebe“
„Kollegen“ und „Kolleginnen“ häufig vorkommen, da so viele Redebeiträge
eröffnet werden. Diese Problematik hat sich zum Beispiel im Topic 6
niedergeschlagen. In vielen Topics spiegeln sich vor allem Themen von bloßer
nationaler Tragweite wieder. Dies lässt sich darin erkennen, dass oft die
einzelnen Parteien erwähnt werden. So wird es zum Beispiel zum Problem, dass
alle Ziffern und Satzzeichen aus dem Korpus entfernt wurden, da die Partei
„Bündnis90/die Grünen“ nicht mehr genau ausgemacht werden kann. Das alleinige
Erscheinen des Wortes „Bündnis“ könnte demnach auf die Partei oder ein
anderes, beliebiges Bündnis hindeuten. Im Falle der vorliegenden Analyse ist
dies ein wirkliches Problem, da das Wort „Bündnis“ auch ein Indikator für das
SDG 17 „Partnerschaften zur Erreichung der Ziele“ sein könnte. Hier trifft man
auf altbekannte Grenzen des Distant Reading: Worte haben abhängig von ihrem
Kontext unterschiedliche Bedeutungen, die in einer rein quantitativen Analyse
nicht verstanden werden können \cite{Ramage}. Die Topics des Korpus der
Bundestagsreden zeigen nur zwei Verbindungen zu den SDGs 3 „Gesundheit und
Wohlergehen“ und 8 „Menschenwürdige Arbeit und Wirtschaftswachstum“.  Ob bei
diesen Topics von einer globalen Tragweite die Rede ist, lässt sich schwer
sagen. Besonders in der eingangs erwähnten pandemischen Lage könnte das
erkannte Topic 4 auch einfach im Bezug auf die Coronapandemie erstellt worden
sein.
%\clearpage

\section{Fazit}
Nach Interpretation der Ergebnisse lässt sich sagen, dass die zu Anfang
geplante Strategie, mittels Topic Modeling von 17 Topics ein gutes Matching
mit den SDGs zu erhalten, nicht aufgegangen ist. Die Topics sind in sich oft
nicht schlüssig und lassen selten eine Zuordnung zu einem der SDGs zu. Als
voreiliges Fazit ließe sich nun sagen, dass die SDGs in deutschen
Bundestagsreden ohne oder nur von geringer von Bedeutung sind. Im Anbetracht
der verwirrenden Ergebnisse des Topic Modelings müsste aber fairerweise auch
davon ausgegangen werden, dass es keine einheitlichen Themenblöcke sind, die
zur Sprache kommen. So bleibt davon auszugehen, dass für eine Beantwortung der
Eingangsfrage an einer anderen Stellschraube gedreht werden muss. Sprich, die
gewählte Methode hat sich nicht als geeignet erwiesen oder das Problem
befindet sich in den Daten. Diese Problematik soll im folgenden Punkt weiter
ausgearbeitet werden.

Prinzipiell ist die Methode des Topic Modeling für so große Mengen von Texten
wie in der durchgeführten Analyse ein probates Mittel. Vielleicht hätte sich
ein gelabeltes LDA Modell, das gleich auch die Überschriften zu den Topic
generiert, besser bewährt, um die Zuordnung zu den Topics nicht vom Ermessen
der Autorin abhängig zu machen. Da jedoch die Ergebnisse des Korpus der
Bundestagsreden weitaus schlechter ausfielen als die des BNE-Korpus, ist es
nicht auszuschließen, dass das verwendete Korpus in dieser Form nicht geeignet
war. So hätte es eventuell mehr an Preprocessing benötigt, um Floskeln,
Redewendungen und Organisatorisches aus dem Datensatz zu entfernen. Sollte die
Forschung weiter fortgeführt werden, wäre es eine Überlegung wert, die Themen
oder Tagesordnungspunkte der Plenartagungen zu verwenden, da hier die Dichte
an Information um einiges höher sein dürfte.

Abschließend bleibt noch darauf hinzuweisen, dass selbst, wenn die SDGs in den
Reden des deutschen Bundestages erwähnt werden und sich mitunter sogar mit der
politischen Agenda decken, ein strukturelles Problem der SDGs weiterhin
bestehen bleibt. Auch wenn die hoch gesteckten Ziele der SDGs als radikal und
nie dagewesen gelobt werden, bezeichnet Aram Ziai in einem Artikel die SDGs
als „Trostpflaster eines Ungleichheit produzierendes globalen Kapitalismus“
\cite{Ziai}.  Ja, durch die ehrgeizigen Ziele der SDG wurden gerade in China
und anderen Ländern des globalen Süden ein Erfolgreicher Kampf gegen die Armut
geführt.  Jedoch lassen sich diese Erfolge nur unter einem
postkolonialistischen Standpunkt feiern. Unter diesem kann Reichtum nur unter
der Prämisse von Aneignung billiger Rohstoffe, Zerstörung von Ökosystemen und
nicht erneuerbarer Ressourcen fußen. Um eine wie eingangs beschriebene sozial,
ökonomisch und ökologisch gerechte Welt zu erlangen, müsste eine Veränderung
hin zu einem Gesellschaftsmodell vollzogen werden, die nicht auf dem
althergebrachten Verständnis von Reichtum basiert \cite{Ziai}. Auch wenn die
SDG ein guter Anfang sind, den -- zumeist -- menschengemachten Problemen der
Menschheit entgegenzuwirken, bleibt doch fraglich, ob dies unter den
bisherigen Rahmenbedingungen gelingen kann und sollte.
%\clearpage

\raggedright
\begin{thebibliography}{xxx}
\bibitem{BMZ} Bundesministerium für wirtschaftliche Zusammenarbeit und
  Entwicklung. 
  \url{https://www.bmz.de/de/service/lexikon/mdg-millenniumsentwicklungsziele-mdgs-14674}, aufgerufen am 20.10.2021.
        
\bibitem{UN} Millenniums-Entwicklungsziele, Bericht 2015. Vereinte Nationen. 
  \url{https://www.un.org/Depts/german/millennium/MDG\%Report\%202015\%20German.pdf}
        
\bibitem{Agenda2030} Bundesministerium für wirtschaftliche Zusammenarbeit und
  Entwicklung, Agenda 2030.
  \url{https://www.bmz.de/de/agenda-2030/sdg-17}. Aufgerufen am 28.10.2021.
        
\bibitem{Bericht} Freiwilliger Staatenbericht Deutschlands zum Hochrangigen
  Politischen Forum für Nachhaltige Entwicklung 2021. Die Bundesregierung,
  Juni 2021.
  \url{https://www.bmz.de/resource/blob/86824/6631843da2eb297d849b03d883140fb7/staatenbericht-deutschlands-zum-hlpf-2021.PDF}
        
\bibitem{Ziai} Aram Ziai. \emph{Die SDGs – Eine postkoloniale Weinprobe.}
  Entwicklungspolitik in Zeiten der SDGs. In: Entwicklungspolitik in Zeiten
  der SDGs. Essays zum 80. Geburtstag von Franz Nuscheler (2018), S. 205-209.
  ISBN: 978-3-939218-47-0.
    
\bibitem{Bundestag} \url{https://www.bundestag.de/services/opendata}
        
\bibitem{OpenDiscourse} F. Richter, P. Koch, O. Franke, J. Kraus, F. Kuruc,
  A. Thiem, J. Högerl, S. Heine, K. Schöps, K. (2020). Open Discourse.
  Harvard Dataverse. V3.  \url{https://doi.org/10.7910/DVN/FIKIBO.}
        
\bibitem{BNE} Bundesministerium für Bildung und Forschung.
  \url{https://www.bne-portal.de/bne/de/einstieg/was-ist-bne/was-ist-bne_node.html},
  aufgerufen am 28.10.2021.
        
\bibitem{Moretti} F. Moretti. Distant Reading. London: Verso (2013).
    
\bibitem{MALLET} A.K. McCallum.  \emph{MALLET: A Machine Learning for Language
  Toolkit.}  \url{http://mallet.cs.umass.edu. 2002.}
    
\bibitem{MALLET_WELSH} 
  \url{https://melaniewalsh.github.io/Intro-Cultural-Analytics/05-Text-Analysis/07-Topic-Modeling-Set-Up.html},
  aufgerufen am 28.10.2021
        
\bibitem{LDA} David M. Blei, Andrew Y. Ng, Michael I. Jordan. \emph{Latent
  dirichlet allocation.} The Journal of Machine Learning Research 3 (2003):
  993-1022.
    
\bibitem{Ramage} Daniel Ramage et al.  \emph{Topic modeling for the social
  sciences.} NIPS 2009 workshop on applications for topic models: text and
  beyond. Vol. 5. 2009.
\end{thebibliography}

\end{document}
